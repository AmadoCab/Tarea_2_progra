\documentclass{article}
\usepackage[utf8]{inputenc}
\usepackage{multicol}
\usepackage{enumerate}
\usepackage{amsmath}
\usepackage{amssymb}
\usepackage{graphicx}
\usepackage{dsfont}
\usepackage[table,xcdraw]{xcolor}
\usepackage{tikz}
\usepackage{float}
\usepackage{array}
\usepackage{mathtools}
\usepackage[spanish]{babel}
\decimalpoint

\usetikzlibrary{arrows.meta, calc}
\setlength{\parindent}{0pt}
\newcommand{\nulls}{\varnothing}

\title{Tarea 2, Progra 2}
\author{Amado Alberto Cabrera Estrada\\
201905757}
\date{March 2021}

\begin{document}

\maketitle

\section*{Jurado}
\subsection*{Ejercicio 3.1}
Describir los lenguajes representados por las siguientes expresiones regulares definidas sobre el alfabeto $\Sigma=\{\texttt{a},\texttt{b},\texttt{c}\}$.
\begin{itemize}
    \item $(\texttt{a}|\texttt{b})^*\texttt{c}$\\
    Las cadenas aceptadas empiezan por cualquier cadena formada por \texttt{a}, \texttt{b} y $\varepsilon$ y acaban en c.
    
    \item $(\texttt{aa}^+)(\texttt{bb}^*)$\\
    Las cadenas aceptadas empiezan por \texttt{a} de dos a infinitas veces y acaban con \texttt{b} de una a infinitas veces.
    
    \item $(\texttt{aa}^+)|(\texttt{bb}^*)$\\
    Las cadenas aceptadas son: \texttt{a} de dos a infinitas veces o \texttt{b} de una a infinitas veces (solo puede ser aceptada una u otra, no ambas a la vez).
    
    \item $\texttt{a}^*\texttt{b}^*\texttt{c}^*$\\
    Las cadenas aceptadas llevan \texttt{a} de cero a infinitas veces, concatenado con \texttt{b} de cero a infinitas veces, concatenado con \texttt{c} de cero a infinitas veces.
\end{itemize}

\subsection*{Ejercicio 3.2}
Representar, mediante una expresión regular, los siguientes lenguajes
\begin{enumerate}
    \item  Considerando que $\Sigma=\{\texttt{a}\}$
    \begin{enumerate}[a)]
        \item el lenguaje formado por cadenas de letras \texttt{a} de longitud par\\
        \texttt{(aa)$^*$}
        
        \item el lenguaje formado por cadenas de letras \texttt{a} de longitud impar\\
        \texttt{a(aa)$^*$}
    \end{enumerate}
    
    \item Considerando que $\Sigma=\{\texttt{a},\texttt{b}\}$ , el lenguaje formado por cadenas de letras a y letras b, de longitud impar, en las que se van alternando los dos símbolos, es decir, nunca aparece el mismo símbolo dos veces seguidas. Por ejemplo: \texttt{abababa} o \texttt{bab}.\\
    \texttt{(b(ab)$^*$)|(a(ba)$^*$)}
\end{enumerate}

\section*{Sipser}
\subsection*{Ejercicio 1.1}
Teniendo los diagramas de estado de dos AFD, $M_1$ y $M_2$. Responda las siguientes preguntas acerca de estas máquinas.
\begin{multicols}{2}
\begin{center}
\begin{tikzpicture}
% Nodes
\node (q1) at (0,0) []{$q_1$};
\node (q2) at (0:2.5) []{$q_2$};
\node (q3) at (-60:2.5) []{$q_3$};
% Circles
\draw[] (q1) circle[radius=0.5];
\draw[] (q2) circle[radius=0.4] circle[radius=0.5];
\draw[] (q3) circle[radius=0.5];
% Edges
\draw[Latex-] ($(q1)+(-0.5,0)$) to[] +(-0.5,0);
\draw[-Latex] ($(q1)+(30:0.5)$) to[bend left] node[auto]{$a$} ($(q2)+(150:0.5)$);
\draw[-Latex] ($(q2)+(-90:0.5)$) to[bend left] node[auto]{$a,b$} ($(q3)+(0:0.5)$);
\draw[Latex-] ($(q2)+(-130:0.5)$) to[bend right] node[auto,swap]{$a$} ($(q3)+(60:0.5)$);
\draw[-Latex] ($(q3)+(180:0.5)$) to[bend left] node[auto]{$b$} ($(q1)+(-90:0.5)$);
\draw[-Latex] ($(q1)+(45:0.5)$) arc[y radius=0.7,x radius=0.35,start angle=0,end angle=180] node[midway,auto,swap]{$b$};
\end{tikzpicture}\\
$M_1$
\end{center}

\begin{center}
\begin{tikzpicture}
% Nodes
\node (q1) at (0,0) []{$q_1$};
\node (q2) at (0:2) []{$q_2$};
\node (q3) at (-90:2) []{$q_3$};
\node (q4) at (2,-2) []{$q_4$};
% Circles
\draw[] (q1) circle[radius=0.4] circle[radius=0.5];
\draw[] (q2) circle[radius=0.5];
\draw[] (q3) circle[radius=0.5];
\draw[] (q4) circle[radius=0.4] circle[radius=0.5];
% Edges
\draw[Latex-] ($(q1)+(-0.5,0)$) to[] +(-0.5,0);
\draw[-Latex] ($(q1)+(45:0.5)$) to[bend left] node[auto]{$b$} ($(q2)+(135:0.5)$);
\draw[-Latex] ($(q2)+(-45:0.5)$) to[bend left] node[auto]{$b$} ($(q4)+(45:0.5)$);
\draw[-Latex] ($(q4)+(225:0.5)$) to[bend left] node[auto]{$a$} ($(q3)+(-45:0.5)$);
\draw[-Latex] ($(q3)+(135:0.5)$) to[bend left] node[auto]{$b$} ($(q1)+(-135:0.5)$);
\draw[-Latex] ($(q3)+(25:0.5)$) to[bend right] node[auto,swap]{$a$} ($(q2)+(-120:0.5)$);
\draw[-Latex] ($(q2)+(205:0.5)$) to[bend right] node[auto,swap]{$a$} ($(q3)+(60:0.5)$);
\draw[-Latex] ($(q1)+(60:0.5)$) arc[x radius=0.25,y radius=0.6,start angle=-15,end angle=195] node[midway,auto,swap]{$a$};
\draw[-Latex] ($(q4)+(-30:0.5)$) arc[y radius=0.25,x radius=0.6,start angle=-90,end angle=90] node[midway,auto,swap]{$b$};
\end{tikzpicture}\\
$M_2$
\end{center}
\end{multicols}
\begin{enumerate}[a.]
    \item ¿Cuál es el estado inicial?\\
    $\boldsymbol{M_1}:$ El estado inicial es $q_1$.\\
    $\boldsymbol{M_2}:$ El estado inicial es $q_1$.
    
    \item ¿Cuál es el conjunto de estados de aceptación\\
    $\boldsymbol{M_1}:$ El conjunto de estados de aceptación es $\{q_2\}$\\
    $\boldsymbol{M_2}:$ El conjunto de estados de aceptación es $\{q_1,q_4\}$
    
    \item ¿Cuál es la secuencia por los que pasa la máquina a través de la cadena \texttt{aabb}?\\
    $\boldsymbol{M_1}:$ La secuencia sería  $q_1\to q_2\to q_3\to q_1\to q_1$\\
    $\boldsymbol{M_2}:$ La secuencia sería  $q_1\to q_1\to q_1\to q_2\to q_4$
    
    \item ¿Acepta la máquina la cadena \texttt{aabb}?\\
    $\boldsymbol{M_1}:$ No, queda en el estado $q_1$ y no es de aceptación.\\
    $\boldsymbol{M_2}:$ Sí, pues acaba en el estado $q_4$.
    
    \item ¿Acepta la máquina la cadena $\boldsymbol{\varepsilon}$?\\
    $\boldsymbol{M_1}:$ No, pues $q_1$ no es de aceptación.\\
    $\boldsymbol{M_2}:$ Sí, $q_1$ es de aceptación.\\
\end{enumerate}

\newpage

\subsection*{Ejercicio 1.3}
La descripción formal de un AFD $M$ es $(\{q_1,\,q_2,q_3,\,q_4,\,q_5\},\ \{u,\,d\},\ \delta,\ q_3,\ \{q_3\})$, donde $\delta$ está dado por la siguiente tabla. Dé el diagrama de estado de esta máquina.
\begin{center}
    \begin{table}[H]
\centering
\resizebox{0.2\textwidth}{!}{
\begin{tabular}{l!{\vrule width 1pt}
>{\columncolor[HTML]{FBFBFB}}l 
>{\columncolor[HTML]{FBFBFB}}l }
& \cellcolor[HTML]{E2E2E2}$\texttt{u}$ & \cellcolor[HTML]{E2E2E2}$\texttt{d}$\\ \noalign{\hrule height 0.9pt}
\cellcolor[HTML]{E2E2E2}$q_1$ & $q_1$                      & $q_2$\\
\cellcolor[HTML]{E2E2E2}$q_2$      & $q_1$                      & $q_3$\\
\cellcolor[HTML]{E2E2E2}$q_3$      & $q_2$                      & $q_4$\\
\cellcolor[HTML]{E2E2E2}$q_4$      & $q_3$                      & $q_5$\\
\cellcolor[HTML]{E2E2E2}$q_5$      & $q_4$                     & $q_5$ \\
\end{tabular}}
\end{table}
    \begin{tikzpicture}
% Nodes
\node (q1) at (0,0) []{$q_1$};
\node (q2) at (2.5,0) []{$q_2$};
\node (q3) at (5,0) []{$q_3$};
\node (q4) at (7.5,0) []{$q_4$};
\node (q5) at (10,0) []{$q_5$};
% Circles
\draw[] (q1) circle[radius=0.5];
\draw[] (q2) circle[radius=0.5];
\draw[] (q3) circle[radius=0.35] circle[radius=0.5];
\draw[] (q4) circle[radius=0.5];
\draw[] (q5) circle[radius=0.5];
% Edges
\draw[Latex-] ($(q3)+(90:0.5)$) to[] +(0,0.5);
\draw[-Latex] ($(q1)+(30:0.5)$) to[bend left] node[auto]{$\texttt{d}$} ($(q2)+(150:0.5)$);
\draw[-Latex] ($(q2)+(30:0.5)$) to[bend left] node[auto]{$\texttt{d}$} ($(q3)+(150:0.5)$);
\draw[-Latex] ($(q3)+(30:0.5)$) to[bend left] node[auto]{$\texttt{d}$} ($(q4)+(150:0.5)$);
\draw[-Latex] ($(q4)+(30:0.5)$) to[bend left] node[auto]{$\texttt{d}$} ($(q5)+(150:0.5)$);
\draw[Latex-] ($(q1)+(-30:0.5)$) to[bend right] node[auto,swap]{$\texttt{u}$} ($(q2)+(-150:0.5)$);
\draw[Latex-] ($(q2)+(-30:0.5)$) to[bend right] node[auto,swap]{$\texttt{u}$} ($(q3)+(-150:0.5)$);
\draw[Latex-] ($(q3)+(-30:0.5)$) to[bend right] node[auto,swap]{$\texttt{u}$} ($(q4)+(-150:0.5)$);
\draw[Latex-] ($(q4)+(-30:0.5)$) to[bend right] node[auto,swap]{$\texttt{u}$} ($(q5)+(-150:0.5)$);
\draw[Latex-] ($(q5)+(60:0.5)$) arc[x radius=0.25,y radius=0.5,start angle=-15,end angle=195] node[midway,auto,swap]{$\texttt{d}$};
\draw[-Latex] ($(q1)+(-60:0.5)$) arc[x radius=0.25,y radius=0.5,start angle=15,end angle=-195] node[midway,auto]{$\texttt{u}$};
\end{tikzpicture}\\
    {\scriptsize parece un elevador}
\end{center}

\newpage

\subsection*{Ejercicio 1.4}
Cada uno de los siguientes lenguajes es la intersección de dos lenguajes más simples. En cada parte construya los AFD para los lenguajes simples y luego combínelos usando la construcción discutida en la nota al pie de página 3 de la página 46 para dar el diagrama de estado del AFD para el lenguaje dado. Para cada uno, $\Sigma=\{\texttt{a},\texttt{b}\}$
\begin{enumerate}[a.]
    \item $\{\omega|\omega\text{ tiene al menos tres letras \texttt{a} y al menos dos letras \texttt{b}}\}$
    \begin{itemize}
        \item $\{\omega|\omega\text{ tiene al menos tres letras \texttt{a}}\}$
        \begin{center}
\begin{tikzpicture}
% Nodes
\node (q0) at (0,0) []{$q_0$};
\node (q1) at (2.5,0) []{$q_1$};
\node (q2) at (5,0) []{$q_2$};
\node (q3) at (7.5,0) []{$q_3$};
% Circles
\draw[] (q0) circle[radius=0.5];
\draw[] (q1) circle[radius=0.5];
\draw[] (q2) circle[radius=0.5];
\draw[] (q3) circle[radius=0.35] circle[radius=0.5];
% Edges
\draw[Latex-] ($(q0)+(180:0.5)$) to[] +(-0.5,0);
\draw[-Latex] ($(q0)+(0:0.5)$) to[] node[auto]{\texttt{a}} ($(q1)+(180:0.5)$);
\draw[-Latex] ($(q1)+(0:0.5)$) to[] node[auto]{\texttt{a}} ($(q2)+(180:0.5)$);
\draw[-Latex] ($(q2)+(0:0.5)$) to[] node[auto]{\texttt{a}} ($(q3)+(180:0.5)$);
\draw[Latex-] ($(q0)+(60:0.5)$) arc[x radius=0.25,y radius=0.5,start angle=-15,end angle=195] node[midway,auto,swap]{\texttt{b}};
\draw[Latex-] ($(q1)+(60:0.5)$) arc[x radius=0.25,y radius=0.5,start angle=-15,end angle=195] node[midway,auto,swap]{\texttt{b}};
\draw[Latex-] ($(q2)+(60:0.5)$) arc[x radius=0.25,y radius=0.5,start angle=-15,end angle=195] node[midway,auto,swap]{\texttt{b}};
\draw[Latex-] ($(q3)+(60:0.5)$) arc[x radius=0.25,y radius=0.5,start angle=-15,end angle=195] node[midway,auto,swap]{\texttt{a,b}};
\end{tikzpicture}
\end{center}
        
        \item $\{\omega|\omega\text{ tiene al menos dos letras \texttt{b}}\}$
        \begin{center}
\begin{tikzpicture}
% Nodes
\node (q0) at (0,0) []{$q_0'$};
\node (q1) at (2.5,0) []{$q_1'$};
\node (q2) at (5,0) []{$q_2'$};
% Circles
\draw[] (q0) circle[radius=0.5];
\draw[] (q1) circle[radius=0.5];
\draw[] (q2) circle[radius=0.35] circle[radius=0.5];
% Edges
\draw[Latex-] ($(q0)+(180:0.5)$) to[] +(-0.5,0);
\draw[-Latex] ($(q0)+(0:0.5)$) to[] node[auto]{\texttt{b}} ($(q1)+(180:0.5)$);
\draw[-Latex] ($(q1)+(0:0.5)$) to[] node[auto]{\texttt{b}} ($(q2)+(180:0.5)$);
\draw[Latex-] ($(q0)+(60:0.5)$) arc[x radius=0.25,y radius=0.5,start angle=-15,end angle=195] node[midway,auto,swap]{\texttt{a}};
\draw[Latex-] ($(q1)+(60:0.5)$) arc[x radius=0.25,y radius=0.5,start angle=-15,end angle=195] node[midway,auto,swap]{\texttt{a}};
\draw[Latex-] ($(q2)+(60:0.5)$) arc[x radius=0.25,y radius=0.5,start angle=-15,end angle=195] node[midway,auto,swap]{\texttt{a,b}};
\end{tikzpicture}
\end{center}
        
        \item Haciendo la tabla
        \begin{table}[H]
\centering
\begin{tabular}{c!{\vrule width 0.8pt}
>{}ccc 
>{}ccc }
$q_0'$ & $(q_0,q_0')$ & $(q_1,q_0')$ & $(q_2,q_0')$ & $(q_3,q_0')$\\
$q_1'$ & $(q_0,q_1')$ & $(q_1,q_1')$ & $(q_2,q_1')$ & $(q_3,q_1')$\\
$q_2'$ & $(q_0,q_2')$ & $(q_1,q_2')$ & $(q_2,q_2')$ & $(q_3,q_2')$\\
\noalign{\hrule height 0.7pt}
\cellcolor[HTML]{E2E2E2}$A\times B$ & $q_0$ & $q_1$ & $q_2$ & $q_3$\\
\end{tabular}
\end{table}
        
        \item Haciendo el autómata
        \begin{center}
\begin{tikzpicture}
% Nodes
\node (q0) at (0:2.8) []{$q_0,q_0'$};
\node (q1) at (30:2.8) []{$q_0,q_1'$};
\node (q2) at (60:2.8) []{$q_0,q_2'$};
\node (q3) at (90:2.8) []{$q_1,q_0'$};
\node (q4) at (120:2.8) []{$q_1,q_1'$};
\node (q5) at (150:2.8) []{$q_1,q_2'$};
\node (q6) at (180:2.8) []{$q_2,q_0'$};
\node (q7) at (210:2.8) []{$q_2,q_1'$};
\node (q8) at (240:2.8) []{$q_2,q_2'$};
\node (q9) at (270:2.8) []{$q_3,q_0'$};
\node (q10) at (300:2.8) []{$q_3,q_1'$};
\node (q11) at (330:2.8) []{$q_3,q_2'$};
% Circles
\draw[] (q0) circle[radius=0.5];
\draw[] (q1) circle[radius=0.5];
\draw[] (q2) circle[radius=0.5];
\draw[] (q3) circle[radius=0.5];
\draw[] (q4) circle[radius=0.5];
\draw[] (q5) circle[radius=0.5];
\draw[] (q6) circle[radius=0.5];
\draw[] (q7) circle[radius=0.5];
\draw[] (q8) circle[radius=0.5];
\draw[] (q9) circle[radius=0.5];
\draw[] (q10) circle[radius=0.5];
\draw[] (q11) circle[radius=0.45] circle[radius=0.5];
% Edges
\draw[Latex-] ($(q0)+(0:0.5)$) to[] +(0.5,0);
\draw[-Latex] ($(q0)+(0:0.5)$) to[] ($(q2)+(0:0.5)$);
\end{tikzpicture}
\end{center}
    \end{itemize}
    
    
    \item $\{\omega|\omega\text{ tiene exactamente dos letras \texttt{a} y al menos dos \texttt{b}}\}$
    \begin{itemize}
        \item $\{\omega|\omega\text{ tiene exactamente dos letras \texttt{a}}\}$
        \input{sipser14/b/b1}
        
        \item $\{\omega|\omega\text{ tiene al menos dos \texttt{b}}\}$
        \input{sipser14/b/b2}
        
        \item Haciendo la tabla
        \begin{table}[H]
\centering
\begin{tabular}{c!{\vrule width 0.8pt}
>{}cc 
>{}cc }
$q_0'$ & $(q_0,q_0')$ & $(q_1,q_0')$ & $(q_2,q_0')$\\
$q_1'$ & $(q_0,q_1')$ & $(q_1,q_1')$ & $(q_2,q_1')$\\
$q_2'$ & $(q_0,q_2')$ & $(q_1,q_2')$ & $(q_2,q_2')$\\
\noalign{\hrule height 0.7pt}
\cellcolor[HTML]{E2E2E2}$A\times B$ & $q_0$ & $q_1$ & $q_2$\\
\end{tabular}
\end{table}
        
        \item Haciendo el autómata
        \input{sipser14/b/br}
    \end{itemize}    
    
    \item $\{\omega|\omega\text{ tiene un número par de letras \texttt{a} y una o dos letras \texttt{b}}\}$
    \begin{itemize}
        \item $\{\omega|\omega\text{ tiene un número par de letras \texttt{a}}\}$
        \begin{center}
\begin{tikzpicture}
% Nodes
\node (q1) at (0,0) []{$q_1$};
\node (q2) at (2.5,0) []{$q_2$};
% Circles
\draw[] (q1) circle[radius=0.35] circle[radius=0.5];
\draw[] (q2) circle[radius=0.5];
% Edges
\draw[Latex-] ($(q0)+(180:0.5)$) to[] +(-0.5,0);
\draw[-Latex] ($(q1)+(30:0.5)$) to[bend left] node[auto]{$\texttt{a}$} ($(q2)+(150:0.5)$);
\draw[Latex-] ($(q1)+(-30:0.5)$) to[bend right] node[auto,swap]{$\texttt{a}$} ($(q2)+(-150:0.5)$);
\draw[Latex-] ($(q2)+(60:0.5)$) arc[x radius=0.25,y radius=0.5,start angle=-15,end angle=195] node[midway,auto,swap]{\texttt{b}};
\draw[Latex-] ($(q1)+(60:0.5)$) arc[x radius=0.25,y radius=0.5,start angle=-15,end angle=195] node[midway,auto,swap]{\texttt{b}};
\end{tikzpicture}
\end{center}
        
        \item $\{\omega|\omega\text{ tiene una o dos letras \texttt{b}}\}$
        \input{sipser14/c/c2}
        
        \item Haciendo la tabla
        \input{sipser14/c/ct}
        
        \item Haciendo el autómata
        \input{sipser14/c/cr}
    \end{itemize}    
    
    \item $\{\omega|\omega\text{ tiene un número par de letras \texttt{a} y cada \texttt{a} se sigue por al menos de una \texttt{b}}\}$
    
    \begin{itemize}
        \item $\{\omega|\omega\text{ tiene un número par de letras \texttt{a}}\}$
        \input{sipser14/d/d1}
        
        \item $\{\omega|\omega\text{ cada \texttt{a} se sigue por al menos de una \texttt{b}}\}$
        \input{sipser14/d/d2}
        
        \item Haciendo la tabla
        \input{sipser14/d/dt}
        
        \item Haciendo el autómata
        \input{sipser14/d/dr}
    \end{itemize}    
    
    \item $\{\omega|\omega \text{ comienza con una \texttt{a} y tiene como máximo una \texttt{b}}\}$
    \begin{itemize}
        \item $\{\omega|\omega\text{ comienza con una \texttt{a}}\}$
        \input{sipser14/e/e1}
        
        \item $\{\omega|\omega\text{ tiene como máximo una \texttt{b}}\}$
        \input{sipser14/e/e2}
        
        \item Haciendo la tabla
        \input{sipser14/e/et}
        
        \item Haciendo el autómata
        \input{sipser14/e/er}
    \end{itemize}    
    
    \item $\{\omega|\omega \text{ tiene un número impar de letras \texttt{a} y termina con una \texttt{b}}\}$
    \begin{itemize}
        \item $\{\omega|\omega\text{ tiene un número impar de letras \texttt{a}}\}$
        \begin{center}
\begin{tikzpicture}
% Nodes
\node (q0) at (0,0) []{$q_0$};
\node (q1) at (2.5,0) []{$q_1$};
% Circles
\draw[] (q0) circle[radius=0.5];
\draw[] (q1) circle[radius=0.35] circle[radius=0.5];
% Edges
\draw[Latex-] ($(q0)+(180:0.5)$) to[] +(-0.5,0);
\draw[-Latex] ($(q0)+(30:0.5)$) to[bend left] node[auto]{$\texttt{a}$} ($(q1)+(150:0.5)$);
\draw[Latex-] ($(q0)+(-30:0.5)$) to[bend right] node[auto,swap]{$\texttt{a}$} ($(q1)+(-150:0.5)$);
\draw[Latex-] ($(q1)+(60:0.5)$) arc[x radius=0.25,y radius=0.5,start angle=-15,end angle=195] node[midway,auto,swap]{\texttt{b}};
\draw[Latex-] ($(q0)+(60:0.5)$) arc[x radius=0.25,y radius=0.5,start angle=-15,end angle=195] node[midway,auto,swap]{\texttt{b}};
\end{tikzpicture}
\end{center}

        
        \item $\{\omega|\omega\text{ termina con una \texttt{b}}\}$
        \input{sipser14/f/f2}
        
        \item Haciendo la tabla
        \input{sipser14/f/ft}
        
        \item Haciendo el autómata
        \input{sipser14/f/fr}
    \end{itemize}    
    
    \item $\{\omega|\omega \text{ tiene longitud par y un número impar de letras \texttt{a}}\}$
    \begin{itemize}
        \item $\{\omega|\omega\text{ tiene longitud par}\}$
        \begin{center}
\begin{tikzpicture}
% Nodes
\node (q0) at (0,0) []{$q_0$};
\node (q1) at (2.5,0) []{$q_1$};
% Circles
\draw[] (q0) circle[radius=0.35] circle[radius=0.5];
\draw[] (q1) circle[radius=0.5];
% Edges
\draw[Latex-] ($(q0)+(180:0.5)$) to[] +(-0.5,0);
\draw[-Latex] ($(q0)+(30:0.5)$) to[bend left] node[auto]{$\texttt{a},\texttt{b}$} ($(q1)+(150:0.5)$);
\draw[Latex-] ($(q0)+(-30:0.5)$) to[bend right] node[auto,swap]{$\texttt{a},\texttt{b}$} ($(q1)+(-150:0.5)$);
\end{tikzpicture}
\end{center}
        
        \item $\{\omega|\omega\text{ tiene un número impar de letras \texttt{a}}\}$
        \input{sipser14/g/g2}
        
        \item Haciendo la tabla
        \input{sipser14/g/gt}
        
        \item Haciendo el autómata
        \input{sipser14/g/gr}
    \end{itemize}    
    
\end{enumerate}

\subsection*{Ejercicio 1.6}
Dar los diagramas de estado de los AFD que reconocen los siguientes lenguajes. Para cada uno el alfabeto es $\{0,1\}$
\begin{enumerate}[a.]
    \item $\{\omega|\omega\text{ empieza con 1 y acaba con 0\}}$
    \begin{center}
\begin{tikzpicture}
% Nodes
\node (q0) at (0,0) []{$q_0$};
\node (q1) at (2.5,0) []{$q_1$};
\node (q2) at (5,0) []{$q_2$};
% Circles
\draw[] (q0) circle[radius=0.5];
\draw[] (q1) circle[radius=0.5];
\draw[] (q2) circle[radius=0.35] circle[radius=0.5];
% Edges
\draw[Latex-] ($(q0)+(180:0.5)$) to[] +(-0.5,0);
\draw[-Latex] ($(q0)+(0:0.5)$) to[] node[auto]{$1$} ($(q1)+(180:0.5)$);
\draw[-Latex] ($(q1)+(30:0.5)$) to[bend left] node[auto]{$0$} ($(q2)+(150:0.5)$);
\draw[Latex-] ($(q1)+(-30:0.5)$) to[bend right] node[auto,swap]{$1$} ($(q2)+(-150:0.5)$);
\draw[Latex-] ($(q1)+(60:0.5)$) arc[x radius=0.25,y radius=0.5,start angle=-15,end angle=195] node[midway,auto,swap]{$1$};
\draw[-Latex] ($(q2)+(60:0.5)$) arc[x radius=0.25,y radius=0.5,start angle=-15,end angle=195] node[midway,auto,swap]{$0$};
\end{tikzpicture}
\end{center}


% ████████╗███████╗  ░█████╗░███╗░░░███╗░█████╗░
% ╚══██╔══╝██╔════╝  ██╔══██╗████╗░████║██╔══██╗
% ░░░██║░░░█████╗░░  ███████║██╔████╔██║██║░░██║
% ░░░██║░░░██╔══╝░░  ██╔══██║██║╚██╔╝██║██║░░██║
% ░░░██║░░░███████╗  ██║░░██║██║░╚═╝░██║╚█████╔╝
% ░░░╚═╝░░░╚══════╝  ╚═╝░░╚═╝╚═╝░░░░░╚═╝░╚════╝░


% ██╗░░░██╗  ██╗░░░██╗░█████╗░  ░█████╗░  ████████╗██╗
% ╚██╗░██╔╝  ╚██╗░██╔╝██╔══██╗  ██╔══██╗  ╚══██╔══╝██║
% ░╚████╔╝░  ░╚████╔╝░██║░░██║  ███████║  ░░░██║░░░██║
% ░░╚██╔╝░░  ░░╚██╔╝░░██║░░██║  ██╔══██║  ░░░██║░░░██║
% ░░░██║░░░  ░░░██║░░░╚█████╔╝  ██║░░██║  ░░░██║░░░██║
% ░░░╚═╝░░░  ░░░╚═╝░░░░╚════╝░  ╚═╝░░╚═╝  ░░░╚═╝░░░╚═╝

% ──────▄▀▄─────▄▀▄
% ─────▄█░░▀▀▀▀▀░░█▄
% ─▄▄──█░░░░░░░░░░░█──▄▄
% █▄▄█─█░░▀░░┬░░▀░░█─█▄▄█

    
    \item $\{\omega|\omega\text{ contiene al menos tres números 1}\}$
    \begin{center}
\begin{tikzpicture}
% Nodes
\node (q0) at (0,0) []{$q_0$};
\node (q1) at (2.5,0) []{$q_1$};
\node (q2) at (5,0) []{$q_2$};
\node (q3) at (7.5,0) []{$q_3$};
% Circles
\draw[] (q0) circle[radius=0.5];
\draw[] (q1) circle[radius=0.5];
\draw[] (q2) circle[radius=0.5];
\draw[] (q3) circle[radius=0.35] circle[radius=0.5];
% Edges
\draw[Latex-] ($(q0)+(180:0.5)$) to[] +(-0.5,0);
\draw[-Latex] ($(q0)+(0:0.5)$) to[] node[auto]{1} ($(q1)+(180:0.5)$);
\draw[-Latex] ($(q1)+(0:0.5)$) to[] node[auto]{1} ($(q2)+(180:0.5)$);
\draw[-Latex] ($(q2)+(0:0.5)$) to[] node[auto]{1} ($(q3)+(180:0.5)$);
\draw[Latex-] ($(q0)+(60:0.5)$) arc[x radius=0.25,y radius=0.5,start angle=-15,end angle=195] node[midway,auto,swap]{0};
\draw[Latex-] ($(q1)+(60:0.5)$) arc[x radius=0.25,y radius=0.5,start angle=-15,end angle=195] node[midway,auto,swap]{0};
\draw[Latex-] ($(q2)+(60:0.5)$) arc[x radius=0.25,y radius=0.5,start angle=-15,end angle=195] node[midway,auto,swap]{0};
\draw[Latex-] ($(q3)+(60:0.5)$) arc[x radius=0.25,y radius=0.5,start angle=-15,end angle=195] node[midway,auto,swap]{0,1};
\end{tikzpicture}
\end{center}
    
    \item $\{\omega|\omega\text{ contiene la subcadena 0101}\}$
    \begin{center}
\begin{tikzpicture}
% Nodes
\node (q0) at (180:2) []{$q_0$};
\node (q1) at (252:2) []{$q_1$};
\node (q2) at (324:2) []{$q_2$};
\node (q3) at (36:2) []{$q_3$};
\node (q4) at (108:2) []{$q_4$};
% Circles
\draw[] (q0) circle[radius=0.5];
\draw[] (q1) circle[radius=0.5];
\draw[] (q2) circle[radius=0.5];
\draw[] (q3) circle[radius=0.5];
\draw[] (q4) circle[radius=0.35] circle[radius=0.5];
% Edges
\draw[Latex-] ($(q0)+(180:0.5)$) to[] +(-0.5,0);
\draw[-Latex] ($(q0)+(-54:0.5)$) to[] node[auto]{0} ($(q1)+(126:0.5)$);
\draw[Latex-] ($(q0)+(-84:0.5)$) to[bend right] node[auto,swap]{0} ($(q1)+(156:0.5)$);
\draw[-Latex] ($(q1)+(18:0.5)$) to[] node[auto,swap]{1} ($(q2)+(198:0.5)$);
\draw[-Latex] ($(q2)+(90:0.5)$) to[] node[auto,swap]{0} ($(q3)+(270:0.5)$);
\draw[-Latex] ($(q3)+(162:0.5)$) to[] node[auto,swap]{1} ($(q4)+(342:0.5)$);
\draw[-Latex] ($(q4)+(150:0.5)$) arc[y radius=0.25,x radius=0.5,start angle=75,end angle=285] node[midway,auto,swap]{0,1};
\draw[-Latex] ($(q0)+(60:0.5)$) arc[x radius=0.25,y radius=0.5,start angle=-15,end angle=195] node[midway,auto,swap]{1};
\draw[-Latex] ($(q3)+(198:0.5)$) to[] node[auto,swap]{$0$} ($(q0)+(18:0.5)$);
\draw[-Latex] ($(q2)+(162:0.5)$) to[] node[auto,swap]{$1$} ($(q0)+(-18:0.5)$);
\end{tikzpicture}
\end{center}
    
    \item $\{\omega|\omega\text{ tiene longitud mayor o igual a 3 y su tercer símbolo es 0}\}$
    \input{sipser16/d}
    
    \item $\{\omega|\omega\text{ empieza con 0 y tiene longitud impar, o empieza con uno y tiene longitud par}\}$
    \begin{center}
\begin{tikzpicture}
% Nodes
\node (q2) at (0,0) []{$q_2$};
\node (q1) at (2.5,0) []{$q_1$};
\node (q0) at (5,0) []{$q_0$};
\node (q3) at (7.5,0) []{$q_3$};
\node (q4) at (10,0) []{$q_4$};
% Circles
\draw[] (q0) circle[radius=0.5];
\draw[] (q1) circle[radius=0.5];
\draw[] (q2) circle[radius=0.35] circle[radius=0.5];
\draw[] (q3) circle[radius=0.35] circle[radius=0.5];
\draw[] (q4) circle[radius=0.5];
% Edges
\draw[Latex-] ($(q0)+(90:0.5)$) to[] +(0,0.5);
\draw[-Latex] ($(q0)+(0:0.5)$) to[] node[auto]{$0$} ($(q3)+(180:0.5)$);
\draw[-Latex] ($(q0)+(180:0.5)$) to[] node[auto,swap]{$1$} ($(q1)+(0:0.5)$);
\draw[-Latex] ($(q1)+(150:0.5)$) to[bend right] node[auto,swap]{0,1} ($(q2)+(30:0.5)$);
\draw[Latex-] ($(q1)+(-150:0.5)$) to[bend left] node[auto]{0,1} ($(q2)+(-30:0.5)$);
\draw[-Latex] ($(q4)+(150:0.5)$) to[bend right] node[auto,swap]{0,1} ($(q3)+(30:0.5)$);
\draw[Latex-] ($(q4)+(-150:0.5)$) to[bend left] node[auto]{0,1} ($(q3)+(-30:0.5)$);
\end{tikzpicture}
\end{center}
    
    \item $\{\omega|\omega\text{ no contiene la subcadena 110}\}$
    \begin{center}
\begin{tikzpicture}
% Nodes
\node (q0) at (180:2) []{$q_0$};
\node (q1) at (270:2) []{$q_1$};
\node (q2) at (0:2) []{$q_2$};
\node (q3) at (90:2) []{$q_3$};
% Circles
\draw[] (q0) circle[radius=0.35] circle[radius=0.5];
\draw[] (q1) circle[radius=0.35] circle[radius=0.5];
\draw[] (q2) circle[radius=0.35] circle[radius=0.5];
\draw[] (q3) circle[radius=0.5];
% Edges
\draw[Latex-] ($(q0)+(180:0.5)$) to[] +(-0.5,0);
\draw[-Latex] ($(q0)+(-75:0.5)$) to[bend right] node[auto,swap]{$1$} ($(q1)+(165:0.5)$);
\draw[Latex-] ($(q0)+(-15:0.5)$) to[bend left] node[auto,swap]{$0$} ($(q1)+(105:0.5)$);
\draw[-Latex] ($(q1)+(45:0.5)$) to[] node[auto,swap]{$1$} ($(q2)+(-135:0.5)$);
\draw[-Latex] ($(q2)+(135:0.5)$) to[] node[auto,swap]{$0$} ($(q3)+(-45:0.5)$);
\draw[-Latex] ($(q3)+(150:0.5)$) arc[y radius=0.25,x radius=0.5,start angle=75,end angle=285] node[midway,auto,swap]{$0,1$};
\draw[Latex-] ($(q2)+(30:0.5)$) arc[y radius=0.25,x radius=0.5,start angle=105,end angle=-105] node[midway,auto]{$1$};
\draw[Latex-] ($(q0)+(60:0.5)$) arc[x radius=0.25,y radius=0.5,start angle=-15,end angle=195] node[midway,auto,swap]{$0$};
\end{tikzpicture}
\end{center}
    
    \item $\{\omega|\text{la longitud de }\omega\text{ es a lo mucho 5}\}$
    \begin{center}
\begin{tikzpicture}
% Nodes
\node (q0) at (0,0) []{$q_0$};
\node (q1) at (0,-2) []{$q_1$};
\node (q2) at (2,0) []{$q_2$};
\node (q4) at (6,-2) []{$q_4$};
\node (q6) at (2,-2) []{$q_6$};
\node (q3) at (4,0) []{$q_3$};
\node (q5) at (4,-2) []{$q_5$};
% Circles
\draw[] (q0) circle[radius=0.35] circle[radius=0.5];
\draw[] (q1) circle[radius=0.35] circle[radius=0.5];
\draw[] (q2) circle[radius=0.35] circle[radius=0.5];
\draw[] (q3) circle[radius=0.35] circle[radius=0.5];
\draw[] (q4) circle[radius=0.35] circle[radius=0.5];
\draw[] (q5) circle[radius=0.35] circle[radius=0.5];
\draw[] (q6) circle[radius=0.5];
% Edges
\draw[Latex-] ($(q0)+(180:0.5)$) to[] +(-0.5,0);
\draw[-Latex] ($(q0)+(-90:0.5)$) to[] node[auto,swap]{$0,1$} ($(q1)+(90:0.5)$);
\draw[-Latex] ($(q1)+(45:0.5)$) to[] node[auto]{$0,1$} ($(q2)+(-135:0.5)$);
\draw[-Latex] ($(q2)+(0:0.5)$) to[] node[auto]{$0,1$} ($(q3)+(180:0.5)$);
\draw[-Latex] ($(q3)+(-45:0.5)$) to[] node[auto]{$0,1$} ($(q4)+(135:0.5)$);
\draw[-Latex] ($(q4)+(180:0.5)$) to[] node[auto]{$0,1$} ($(q5)+(0:0.5)$);
\draw[-Latex] ($(q5)+(180:0.5)$) to[] node[auto]{$0,1$} ($(q6)+(0:0.5)$);
\draw[-Latex] ($(q6)+(-120:0.5)$) arc[x radius=0.25,y radius=0.5,start angle=180,end angle=360] node[midway,auto,swap]{$0,1$};
\end{tikzpicture}\\
{\scriptsize Parece una tortuguita.}
\end{center}
    
    \item $\{\omega|\omega\text{ es cualquier cadena excepto 11 y 111}\}$
    \begin{center}
\begin{tikzpicture}
% Nodes
\node (q4) at (0:0) []{$q_4$};
\node (q3) at (180:2) []{$q_3$};
\node (q2) at (270:2) []{$q_2$};
\node (q1) at (0:2) []{$q_1$};
\node (q0) at (90:2) []{$q_0$};
% Circles
\draw[] (q4) circle[radius=0.35] circle[radius=0.5];
\draw[] (q3) circle[radius=0.5];
\draw[] (q2) circle[radius=0.5];
\draw[] (q1) circle[radius=0.35] circle[radius=0.5];
\draw[] (q0) circle[radius=0.35] circle[radius=0.5];
% Edges
\draw[Latex-] ($(q0)+(90:0.5)$) to[] +(0,0.5);
\draw[Latex-] ($(q3)+(-45:0.5)$) to[] node[auto,swap]{$1$} ($(q2)+(135:0.5)$);
\draw[Latex-] ($(q2)+(45:0.5)$) to[] node[auto,swap]{$1$} ($(q1)+(-135:0.5)$);
\draw[Latex-] ($(q1)+(135:0.5)$) to[] node[auto,swap]{$1$} ($(q0)+(-45:0.5)$);

\draw[-Latex] ($(q0)+(-90:0.5)$) to[] node[auto,swap]{$0$} ($(q4)+(90:0.5)$);
\draw[-Latex] ($(q1)+(0:-0.5)$) to[] node[auto,swap]{$0$} ($(q4)+(180:-0.5)$);
\draw[-Latex] ($(q2)+(90:0.5)$) to[] node[auto,swap]{$0$} ($(q4)+(-90:0.5)$);
\draw[-Latex] ($(q3)+(180:-0.5)$) to[] node[auto,swap]{$0,1$} ($(q4)+(0:-0.5)$);

\draw[Latex-,rotate=-45] ($(q4)+(150:0.5)$) arc[y radius=0.25,x radius=0.5,start angle=75,end angle=285] node[midway,auto,swap]{$0,1$};
\end{tikzpicture}
\end{center}
    
    \item $\{\omega|\text{cualquier posición impar de }\omega\text{ es un 1}\}$
    \begin{center}
\begin{tikzpicture}
% Nodes
\node (q0) at (0,0) []{$q_0$};
\node (q1) at (2.5,0) []{$q_1$};
\node (q2) at (5,0) []{$q_2$};
\node (q3) at (2.5,2) []{$q_3$};
% Circles
\draw[] (q0) circle[radius=0.5];
\draw[] (q1) circle[radius=0.35] circle[radius=0.5];
\draw[] (q2) circle[radius=0.35] circle[radius=0.5];
\draw[] (q3) circle[radius=0.5];
% Edges
\draw[Latex-] ($(q0)+(180:0.5)$) to[] +(-0.5,0);
\draw[-Latex] ($(q0)+(0:0.5)$) to[] node[auto,swap]{$1$} ($(q1)+(180:0.5)$);
\draw[-Latex] ($(q1)+(30:0.5)$) to[bend left] node[auto]{$0,1$} ($(q2)+(150:0.5)$);
\draw[Latex-] ($(q1)+(-30:0.5)$) to[bend right] node[auto]{$1$} ($(q2)+(-150:0.5)$);
\draw[-Latex] ($(q0)+(90:0.5)$) to[bend left] node[auto]{$0$} ($(q3)+(180:0.5)$);
\draw[-Latex] ($(q2)+(90:0.5)$) to[bend right] node[auto,swap]{$0$} ($(q3)+(0:0.5)$);
\draw[-Latex] ($(q3)+(60:0.5)$) arc[x radius=0.25,y radius=0.5,start angle=0,end angle=180] node[midway,auto,swap]{$0,1$};
\end{tikzpicture}
\end{center}
    
    \item $\{\omega|\omega\text{ contiene al menos dos números 0 y como mucho un 1}\}$
    \begin{center}
\begin{tikzpicture}
% Nodes
\node (q0) at (0,0) []{$q_0$};
\node (q1) at (2,0) []{$q_1$};
\node (q2) at (4,0) []{$q_2$};
\node (q3) at (6,0) []{$q_3$};
\node (q4) at (2,-2) []{$q_4$};
\node (q5) at (2,-4) []{$q_5$};
\node (q6) at (0,-2) []{$q_6$};
\node (q7) at (0,-4) []{$q_7$};
\node (q8) at (0,-6) []{$q_8$};
% Circles
\draw[] (q0) circle[radius=0.5];
\draw[] (q1) circle[radius=0.5];
\draw[] (q2) circle[radius=0.35] circle[radius=0.5];
\draw[] (q3) circle[radius=0.35] circle[radius=0.5];
\draw[] (q4) circle[radius=0.5];
\draw[] (q5) circle[radius=0.35] circle[radius=0.5];
\draw[] (q6) circle[radius=0.5];
\draw[] (q7) circle[radius=0.5];
\draw[] (q8) circle[radius=0.35] circle[radius=0.5];
% Edges
\draw[Latex-] ($(q0)+(180:0.5)$) to[] +(-0.5,0);

\draw[-Latex] ($(q0)+(0:0.5)$) to[] node[auto]{0} ($(q1)+(180:0.5)$);
\draw[-Latex] ($(q1)+(0:0.5)$) to[] node[auto]{0} ($(q2)+(180:0.5)$);
\draw[-Latex] ($(q2)+(0:0.5)$) to[] node[auto]{1} ($(q3)+(180:0.5)$);
\draw[-Latex] ($(q1)+(-90:0.5)$) to[] node[auto]{1} ($(q4)+(90:0.5)$);
\draw[-Latex] ($(q4)+(-90:0.5)$) to[] node[auto]{0} ($(q5)+(90:0.5)$);
\draw[-Latex] ($(q0)+(-90:0.5)$) to[] node[auto]{1} ($(q6)+(90:0.5)$);
\draw[-Latex] ($(q6)+(-90:0.5)$) to[] node[auto]{0} ($(q7)+(90:0.5)$);
\draw[-Latex] ($(q7)+(-90:0.5)$) to[] node[auto]{0} ($(q8)+(90:0.5)$);

\draw[Latex-] ($(q2)+(60:0.5)$) arc[x radius=0.25,y radius=0.5,start angle=-15,end angle=195] node[midway,auto,swap]{0};
\draw[Latex-] ($(q3)+(60:0.5)$) arc[x radius=0.25,y radius=0.5,start angle=-15,end angle=195] node[midway,auto,swap]{0};
\draw[-Latex] ($(q8)+(150:0.5)$) arc[y radius=0.25,x radius=0.5,start angle=75,end angle=285] node[midway,auto,swap]{0};
\draw[Latex-, rotate=180] ($(q5)+(150:0.5)$) arc[y radius=0.25,x radius=0.5,start angle=75,end angle=285] node[midway,auto,swap]{0};
\end{tikzpicture}
\end{center}
    
    \item $\{\varepsilon,0\}$
    \begin{center}
\begin{tikzpicture}
% Nodes
\node (q0) at (0,0) []{$q_0$};
\node (q1) at (2.5,0) []{$q_1$};
\node (q2) at (60:2.5) []{$q_2$};
% Circles
\draw[] (q0) circle[radius=0.35] circle[radius=0.5];
\draw[] (q1) circle[radius=0.5];
\draw[] (q2) circle[radius=0.35] circle[radius=0.5];
% Edges
\draw[Latex-] ($(q0)+(180:0.5)$) to[] +(-0.5,0);
\draw[-Latex] ($(q0)+(60:0.5)$) to[] node[auto]{$0$} ($(q2)+(-120:0.5)$);
\draw[-Latex] ($(q0)+(0:0.5)$) to[] node[auto,swap]{$1$} ($(q1)+(180:0.5)$);
\draw[-Latex] ($(q2)+(-60:0.5)$) to[] node[auto]{$0,1$} ($(q1)+(120:0.5)$);
\draw[-Latex] ($(q1)+(30:0.5)$) arc[y radius=0.25,x radius=0.5,start angle=100,end angle=-100] node[midway,auto]{$0,1$};
\end{tikzpicture}
\end{center}
    
    \item $\{\omega|\omega\text{ contiene un número par de números 0, o contiene exactamente dos números 1}\}$
    \begin{center}
\begin{tikzpicture}
% Nodes
\node (q0) at (0,0) []{$q_0$};
\node (q1) at (3,0) []{$q_1$};
\node (q2) at (3,3) []{$q_2$};
\node (q3) at (0,3) []{$q_3$};
% Circles
\draw[] (q0) circle[radius=0.35] circle[radius=0.5];
\draw[] (q1) circle[radius=0.5];
\draw[] (q2) circle[radius=0.5];
\draw[] (q3) circle[radius=0.35] circle[radius=0.5];
% Edges
\draw[Latex-] ($(q0)+(180:0.5)$) to[] +(-0.5,0);
\draw[-Latex] ($(q0)+(15:0.5)$) to[bend left] node[auto]{$0$} ($(q1)+(165:0.5)$);
\draw[Latex-] ($(q0)+(-15:0.5)$) to[bend right] node[auto,swap]{$0$} ($(q1)+(-165:0.5)$);
\draw[-Latex,rotate=90] ($(q3)+(60:0.5)$) arc[x radius=0.25,y radius=0.5,start angle=0,end angle=180] node[midway,auto,swap]{$0$};
\draw[-Latex] ($(q1)+(90:0.5)$) to[] node[auto,swap]{$1$} ($(q2)+(-90:0.5)$);
\draw[-Latex] ($(q2)+(180:0.5)$) to[] node[auto,swap]{$1$} ($(q3)+(0:0.5)$);
\end{tikzpicture}
\end{center}
    
    \item El conjunto vacío.
    \begin{center}
\begin{tikzpicture}
% Nodes
\node (q0) at (0,0) []{$q_0$};
% Circles
\draw[] (q0) circle[radius=0.5];
% Edges
\draw[Latex-] ($(q0)+(180:0.5)$) to[] +(-0.5,0);
\draw[-Latex] ($(q0)+(60:0.5)$) arc[x radius=0.25,y radius=0.5,start angle=0,end angle=180] node[midway,auto,swap]{$0,1$};
\end{tikzpicture}
\end{center}
    
    \item Todas las cadenas menos la cadena vacía
    \input{sipser16/n}
\end{enumerate}

\subsection*{Ejercicio 1.8}
Use la construcción en la demostración del teorema 1.45 para dar los diagramas de estado de los AFND reconociendo la unión de los lenguajes descritos en
\begin{enumerate}[a.]
    \item Ejercicios 1.6a y 1.6b.
    
    \item Ejercicios 1.6c y 1.6f.
\end{enumerate}

\subsection*{Ejercicio 1.11}
Demuestre que todo AFND puede ser convertido a uno equivalente que tenga un solo estado de aceptación.\vspace{0.5em}\\
Teniendo un AFND $N=\{Q,\Sigma,\delta,s,F\}$ tal que la cardinalidad de $F$ es mayor a 1. Construiremos un nuevo AFND $N'$ tal que $N'=\{Q',\Sigma',\delta',s',F'\}$ en donde:
\begin{itemize}
    \item $Q'=Q\cup\{q_a\}$
    
    \item $\Sigma=\Sigma'$
    
    \item $\delta'(q,r)=\left\{
    \begin{array}{ll}
        \delta(q,r) & q\not\in F\text{ y }r\not=\epsilon\\
        q_a & q\in F\text{ y }r=\epsilon
    \end{array}
    \right.$
    
    \item $s'=s$
    
    \item $F'=\{q_a\}$
\end{itemize}
Esto quiere decir que se conecta cada estado aceptado por $N$ con un nuevo estado único de aceptación $\{q_a\}$ y se puede llegar a él a partir de una $\varepsilon$–transición, por lo tanto se unifican todos los estados de aceptación en uno solo. Similar a lo que hacemos para transformar un autómata a una expresión regular.

\subsection*{Ejercicio 1.12}
Sea $D = \{w|w$ contiene un número par de letras \texttt{a}, un número impar de letras \texttt{b} y no contiene la subcadena \texttt{ab}$\}$. Dé un AFD con cinco estados que reconozca $D$ y una expresión regular que genere a $D$. (Sugerencia: Describa $D$ de manera más simple).
% \begin{center}
% \begin{tikzpicture}
% % Nodes
% \node (q4) at (180:2) []{$q_4$};
% \node (q3) at (252:2) []{$q_3$};
% \node (q2) at (324:2) []{$q_2$};
% \node (q1) at (36:2) []{$q_1$};
% \node (q0) at (108:2) []{$q_0$};
% % Circles
% \draw[] (q4) circle[radius=0.5];
% \draw[] (q3) circle[radius=0.5];
% \draw[] (q2) circle[radius=0.5];
% \draw[] (q1) circle[radius=0.5] circle[radius=0.35];
% \draw[] (q0) circle[radius=0.5];
% % Edges
% \draw[Latex-] ($(q4)+(180:0.5)$) to[] +(-0.5,0);

% \draw[Latex-, rotate=45] ($(q0)+(-84:0.5)$) to[bend right] node[auto,swap]{\texttt{b}} ($(q1)+(140:0.5)$);
% \draw[-Latex] ($(q0)+(5:0.5)$) to[bend left] node[auto]{\texttt{b}} ($(q1)+(130:0.5)$);

% \draw[-Latex, rotate=-50] ($(q1)+(-10:0.5)$) to[bend left] node[auto]{\texttt{a}} ($(q2)+(115:0.5)$);
% \draw[-Latex] ($(q2)+(5:0.5)$) to[bend left] node[auto]{\texttt{b}} ($(q1)+(130:0.5)$);


% \draw[-Latex] ($(q4)+(-54:0.5)$) to[] node[auto]{0} ($(q3)+(126:0.5)$);
% \draw[Latex-] ($(q4)+(-84:0.5)$) to[bend right] node[auto,swap]{0} ($(q3)+(156:0.5)$);
% \draw[-Latex] ($(q3)+(18:0.5)$) to[] node[auto,swap]{1} ($(q2)+(198:0.5)$);
% \draw[-Latex] ($(q2)+(90:0.5)$) to[] node[auto,swap]{0} ($(q1)+(270:0.5)$);
% % \draw[-Latex] ($(q1)+(162:0.5)$) to[] node[auto,swap]{1} ($(q0)+(342:0.5)$);

% \draw[-Latex] ($(q0)+(150:0.5)$) arc[y radius=0.25,x radius=0.5,start angle=75,end angle=285] node[midway,auto,swap]{0,1};
% \draw[-Latex] ($(q4)+(60:0.5)$) arc[x radius=0.25,y radius=0.5,start angle=-15,end angle=195] node[midway,auto,swap]{1};
% \draw[-Latex] ($(q1)+(198:0.5)$) to[] node[auto,swap]{$0$} ($(q4)+(18:0.5)$);
% \draw[-Latex] ($(q2)+(162:0.5)$) to[] node[auto,swap]{$1$} ($(q4)+(-18:0.5)$);
% \end{tikzpicture}
% \end{center}

\begin{center}
\begin{tikzpicture}
% Nodes
\node (q2) at (0,0) []{$q_2$};
\node (q1) at (2.5,0) []{$q_1$};
\node (q0) at (5,0) []{$q_0$};
\node (q3) at (7.5,0) []{$q_3$};
\node (q4) at (10,0) []{$q_4$};
% Circles
\draw[] (q0) circle[radius=0.5];
\draw[] (q1) circle[radius=0.5];
\draw[] (q2) circle[radius=0.5];
\draw[] (q3) circle[radius=0.5] circle[radius=0.35];
\draw[] (q4) circle[radius=0.5];
% Edges
\draw[Latex-] ($(q0)+(90:0.5)$) to[] +(0,0.5);
\draw[-Latex] ($(q1)+(180:0.5)$) to[] node[auto,swap]{$\texttt{b}$} ($(q2)+(0:0.5)$);
\draw[-Latex] ($(q0)+(150:0.5)$) to[bend right] node[auto,swap]{\texttt{a}} ($(q1)+(30:0.5)$);
\draw[Latex-] ($(q0)+(-150:0.5)$) to[bend left] node[auto]{\texttt{a}} ($(q1)+(-30:0.5)$);

\draw[-Latex] ($(q3)+(150:0.5)$) to[bend right] node[auto,swap]{\texttt{b}} ($(q0)+(30:0.5)$);
\draw[Latex-] ($(q3)+(-150:0.5)$) to[bend left] node[auto]{\texttt{b}} ($(q0)+(-30:0.5)$);

\draw[-Latex] ($(q4)+(150:0.5)$) to[bend right] node[auto,swap]{\texttt{a}} ($(q3)+(30:0.5)$);
\draw[Latex-] ($(q4)+(-150:0.5)$) to[bend left] node[auto]{\texttt{a}} ($(q3)+(-30:0.5)$);
\end{tikzpicture}
\end{center}

\subsection*{Ejercicio 1.14}
\begin{enumerate}[a)]
    \item Demuestre que si $M$ es un AFD que reconoce el lenguaje $B$, intercambiando el estado de aceptación y de no aceptación en $M$ produce un nuevo AFD reconociendo el complemento de $B$. Concluya que la clase de lenguajes regulares es cerrado bajo complemento.\vspace{0.5em}\\
    Teniendo un AFD $M=\{Q,\Sigma,\delta,s,F\}$ tal que $L(M)=B$ entonces intercambiar los estados de aceptación por no aceptación y viceversa significa que todas las cadenas no aceptadas ahora lo serán, formando el lenguaje $A$. Tal que $\Sigma^*=A\cup B$ y $A\cap B=\varnothing$, esto significa que $A=B^c$. Lo que nos daría un nuevo autómata $M'=\{Q',\Sigma',\delta',s',F'\}$ tal que:
    \begin{itemize}
        \item $Q'=Q$
        \item $\Sigma'=\Sigma$
        \item $\delta'=\delta$
        \item $s'=s$
        \item $F'=Q\backslash F$
    \end{itemize}
    
    \item Demuestre dando un ejemplo de que si $M$ es un AFND que reconoce al lenguaje $C$, intercambiando los estados de aceptación y no aceptación en $M$ no necesariamente produce un nuevo AFND que reconozca el complemento de $C$. ¿Es la clase de lenguajes reconocidos por por ANFD cerrado bajo complemento? Explique su respuesta.\vspace{0.5em}\\
\end{enumerate}

\subsection*{Ejercicio 1.16}
Use la construcción dada en el teorema 1.39 para convertir el siguiente autómata finito no determinista a su equivalente autómata finito determinista.
\begin{multicols}{2}
\begin{center}
\begin{tikzpicture}
% Nodes
\node (q1) at (0,0) []{$1$};
\node (q2) at (-90:2.5) []{$2$};
% Circles
\draw[] (q1) circle[radius=0.35] circle[radius=0.5];
\draw[] (q2) circle[radius=0.5];
% Edges
\draw[Latex-] ($(q1)+(-0.5,0)$) to[] +(-0.5,0);
\draw[-Latex] ($(q1)+(-60:0.5)$) to[bend left] node[auto]{$a,b$}
($(q2)+(60:0.5)$);
\draw[Latex-] ($(q1)+(-120:0.5)$) to[bend right] node[auto,swap]{$b$}
($(q2)+(120:0.5)$);
\draw[-Latex] ($(q1)+(-30:0.5)$) arc[y radius=0.25,x radius=0.5,start angle=-105,end angle=105] node[midway,auto,swap]{$a$};
\end{tikzpicture}\\
(a)\phantom{M}
\end{center}

\begin{center}
\begin{tikzpicture}
% Nodes
\node (q1) at (0,0) []{$1$};
\node (q2) at (0:2.5) []{$2$};
\node (q3) at (-60:2.5) []{$3$};
% Circles
\draw[] (q1) circle[radius=0.5];
\draw[] (q2) circle[radius=0.35] circle[radius=0.5];
\draw[] (q3) circle[radius=0.5];
% Edges
\draw[Latex-] ($(q1)+(-0.5,0)$) to[] +(-0.5,0);
\draw[-Latex] ($(q1)+(30:0.5)$) to[bend left] node[auto]{$\varepsilon$} ($(q2)+(150:0.5)$);
\draw[Latex-] ($(q1)+(-30:0.5)$) to[bend right] node[auto,swap]{$a$} ($(q2)+(210:0.5)$);
\draw[-Latex] ($(q1)+(-60:0.5)$) to[] node[auto,swap]{$a$} ($(q3)+(120:0.5)$);
\draw[-Latex] ($(q3)+(60:0.5)$) to[] node[auto,swap]{$a,b$} ($(q2)+(240:0.5)$);
\draw[-Latex] ($(q3)+(-30:0.5)$) arc[y radius=0.25,x radius=0.5,start angle=-105,end angle=105] node[midway,auto,swap]{$b$};
\end{tikzpicture}\\
\phantom{Mx}(b)
\end{center}
\end{multicols}

\subsection*{Ejercicio 1.17}
\begin{enumerate}[a.]
    \item Dé un ANFD reconociendo el lenguaje (\texttt{01}\,$\cup$\,\texttt{001}\,$\cup$\,\texttt{010})$^*$.
    
    \item Convierta este ANFD a un AFD equivalente. Dé únicamente la porción del AFD que es alcanzable desde el estado inicial.
\end{enumerate}

\subsection*{Ejercicio 1.19}
Use el procedimiento descrito en el Lema 1.55 para convertir la siguiente expresión regular a un autómata finito no determinista.
\begin{enumerate}[a.]
    \item (\texttt{0}\,$\cup$\,\texttt{1})$^*$\texttt{000}(\texttt{0}\,$\cup$\,\texttt{1})$^*$
    
    \item (((\texttt{00})$^*$(\texttt{11}))\,$\cup$\,\texttt{01})$^*$
    
    \item $\varnothing^*$
\end{enumerate}

\subsection*{Ejercicio 1.21}
Use el procedimiento descrito en el Lema 1.60 para convertir el siguiente autómata finito a expresiones regulares.
\begin{multicols}{2}
\begin{center}
\begin{tikzpicture}
% Nodes
\node (q1) at (0,0) []{$1$};
\node (q2) at (-90:2.5) []{$2$};
% Circles
\draw[] (q1) 
    circle[radius=0.5];
\draw[] (q2) 
    circle[radius=0.35] 
    circle[radius=0.5];
% Edges
\draw[Latex-] ($(q1)+(-0.5,0)$) to[] +(-0.5,0);
\draw[-Latex] ($(q1)+(-60:0.5)$) to[bend left] node[auto]{\text{b}}
($(q2)+(60:0.5)$);
\draw[Latex-] ($(q1)+(-120:0.5)$) to[bend right] node[auto,swap]{\text{b}}
($(q2)+(120:0.5)$);
\draw[-Latex] ($(q1)+(30:0.5)$) arc[y radius=0.25,x radius=0.5,start angle=105,end angle=-105] node[midway,auto]{\text{a}};
\draw[Latex-] ($(q2)+(-30:0.5)$) arc[y radius=0.25,x radius=0.5,start angle=-105,end angle=105] node[midway,auto,swap]{\text{a}};
\end{tikzpicture}\\
(a)\phantom{M}
\end{center}

\begin{center}
\begin{tikzpicture}
% Nodes
\node (q1) at (0,0) []{$1$};
\node (q2) at (0:2.5) []{$2$};
\node (q3) at (-60:2.5) []{$3$};
% Circles
\draw[] (q1) circle[radius=0.5];
\draw[] (q2) circle[radius=0.35] circle[radius=0.5];
\draw[] (q3) circle[radius=0.5];
% Edges
\draw[Latex-] ($(q1)+(-0.5,0)$) to[] +(-0.5,0);
\draw[-Latex] ($(q1)+(30:0.5)$) to[bend left] node[auto]{$\varepsilon$} ($(q2)+(150:0.5)$);
\draw[Latex-] ($(q1)+(-30:0.5)$) to[bend right] node[auto,swap]{$a$} ($(q2)+(210:0.5)$);
\draw[-Latex] ($(q1)+(-60:0.5)$) to[] node[auto,swap]{$a$} ($(q3)+(120:0.5)$);
\draw[-Latex] ($(q3)+(60:0.5)$) to[] node[auto,swap]{$a,b$} ($(q2)+(240:0.5)$);
\draw[-Latex] ($(q3)+(-30:0.5)$) arc[y radius=0.25,x radius=0.5,start angle=-105,end angle=105] node[midway,auto,swap]{$b$};
\end{tikzpicture}\\
(b)\phantom{M}
\end{center}
\end{multicols}
\begin{enumerate}[a)]
    \item Primer grafo
    \begin{itemize}
        \item Añadiendo el estado inicial y final.
        \begin{center}
\begin{tikzpicture}
% Nodes
\node (qe) at (0,0) []{$1$};
\node (qo) at (2.5,0) []{$2$};
\node (qs) at (-2.5,0) []{$q_{\text{s}}$};
\node (qa) at (5,0) []{$q_{\text{a}}$};
% Circles
\draw[] (qs) circle[radius=0.5];
\draw[] (qe) circle[radius=0.5];
\draw[] (qa) circle[radius=0.4] circle[radius=0.5];
\draw[] (qo) circle[radius=0.5];
% Edges
\draw[Latex-] ($(qs)+(-0.5,0)$) to[] +(-0.5,0);
\draw[-Latex] ($(qs)+(0:0.5)$) to[] node[auto]{$\varepsilon$} ($(qe)+(180:0.5)$);
\draw[-Latex] ($(qe)+(30:0.5)$) to[bend left] node[auto]{\texttt{b}} ($(qo)+(150:0.5)$);
\draw[-Latex] ($(qo)+(210:0.5)$) to[bend left] node[auto]{\texttt{b}} ($(qe)+(-30:0.5)$);
\draw[-Latex] ($(qe)+(60:0.5)$) arc[x radius=0.25,y radius=0.5,start angle=-15,end angle=195] node[auto,swap,midway]{\texttt{a}};
\draw[-Latex] ($(qo)+(60:0.5)$) arc[x radius=0.25,y radius=0.5,start angle=-15,end angle=195] node[auto,swap,midway]{\texttt{a}};
\draw[-Latex] ($(qo)+(0:0.5)$) to[] node[auto]{$\varepsilon$} ($(qa)+(180:0.5)$);
\end{tikzpicture}
\end{center}
        \item Tomando $q_{\text{rip}}=1$.
        \begin{center}
\begin{tikzpicture}
% Nodes
\node (qo) at (2.5,0) []{$2$};
\node (qs) at (0,0) []{$q_{\text{s}}$};
\node (qa) at (5,0) []{$q_{\text{a}}$};
% Circles
\draw[] (qs) circle[radius=0.5];
\draw[] (qa) circle[radius=0.4] circle[radius=0.5];
\draw[] (qo) circle[radius=0.5];
% Edges
\draw[Latex-] ($(qs)+(-0.5,0)$) to[] +(-0.5,0);
\draw[-Latex] ($(qs)+(0:0.5)$) to[] node[auto]{$(\texttt{a}^*\texttt{b})$} ($(qo)+(180:0.5)$);
\draw[-Latex] ($(qo)+(60:0.5)$) arc[x radius=0.25,y radius=0.5,start angle=-15,end angle=195] node[auto,swap,midway]{$\texttt{a}|(\texttt{b}\texttt{a}^*\texttt{b})$};
\draw[-Latex] ($(qo)+(0:0.5)$) to[] node[auto]{$\varepsilon$} ($(qa)+(180:0.5)$);
\end{tikzpicture}
\end{center}
        \item Tomando $q_{\text{rip}}=2$.
        \begin{center}
\begin{tikzpicture}
% Nodes
\node (qs) at (0,0) []{$q_{\text{s}}$};
\node (qa) at (5,0) []{$q_{\text{a}}$};
% Circles
\draw[] (qs) circle[radius=0.5];
\draw[] (qa) circle[radius=0.4] circle[radius=0.5];
% Edges
\draw[Latex-] ($(qs)+(-0.5,0)$) to[] +(-0.5,0);
\draw[-Latex] ($(qs)+(0:0.5)$) to[] node[auto]{$(\texttt{a}^*\texttt{b})(\texttt{a}|(\texttt{b}\texttt{a}^*\texttt{b}))^*$} ($(qa)+(180:0.5)$);
\end{tikzpicture}
\end{center}
    \end{itemize}
    
    \item Segundo grafo
    \begin{itemize}
        \item Añadiendo estado inicial y final
        \begin{center}
\begin{tikzpicture}
% Nodes
\node (qs) at (60:2.5) []{$q_s$};
\node (q1) at (0,0) []{$1$};
\node (q2) at (0:2.5) []{$2$};
\node (q3) at (-60:2.5) []{$3$};
\node (qa) at (-120:2.5) []{$q_a$};
% Circles
\draw[] (qs) circle[radius=0.5];
\draw[] (q1) circle[radius=0.5];
\draw[] (q2) circle[radius=0.5];
\draw[] (q3) circle[radius=0.5];
\draw[] (qa) circle[radius=0.35] circle[radius=0.5];
% Edges
\draw[Latex-] ($(qs)+(-0.5,0)$) to[] +(-0.5,0);
\draw[-Latex] ($(qs)+(-120:0.5)$) to[] node[auto,swap]{$\varepsilon$} ($(q1)+(60:0.5)$);
\draw[-Latex] ($(q1)+(0:0.5)$) to[] node[auto]{\texttt{a},\texttt{b}} ($(q2)+(180:0.5)$);
\draw[-Latex] ($(q3)+(85:0.5)$) to[bend left] node[auto]{\texttt{b}} ($(q2)+(225:0.5)$);
\draw[Latex-] ($(q1)+(-60:0.5)$) to[] node[auto,swap]{\texttt{a}} ($(q3)+(120:0.5)$);
\draw[Latex-] ($(q3)+(30:0.5)$) to[bend right] node[auto,swap]{\texttt{b}} ($(q2)+(270:0.5)$);
\draw[Latex-] ($(q2)+(-30:0.5)$) arc[y radius=0.25,x radius=0.5,start angle=-105,end angle=105] node[midway,auto,swap]{\texttt{a}};
\draw[-Latex] ($(q1)+(-120:0.5)$) to[] node[auto,swap]{$\varepsilon$} ($(qa)+(60:0.5)$);
\draw[-Latex] ($(q3)+(-180:0.5)$) to[] node[auto]{$\varepsilon$} ($(qa)+(0:0.5)$);
\end{tikzpicture}
\end{center}
        \item Tomando $q_{\text{rip}}=3$.
        \begin{center}
\begin{tikzpicture}[rotate=120]
% Nodes
\node (qs) at (60:2.5) []{$q_s$};
\node (q1) at (0,0) []{$1$};
\node (q2) at (0:2.5) []{$2$};
\node (qa) at (-120:2.5) []{$q_a$};
% Circles
\draw[] (qs) circle[radius=0.5];
\draw[] (q1) circle[radius=0.5];
\draw[] (q2) circle[radius=0.5];
\draw[] (qa) circle[radius=0.35] circle[radius=0.5];
% Edges
\draw[Latex-] ($(qs)+(60:0.5)$) to[] +(60:0.5);
\draw[-Latex] ($(qs)+(-120:0.5)$) to[] node[auto,swap]{$\varepsilon$} ($(q1)+(60:0.5)$);
\draw[-Latex] ($(q1)+(30:0.5)$) to[bend left] node[auto]{\texttt{a,b}} ($(q2)+(150:0.5)$);
\draw[Latex-] ($(q1)+(-30:0.5)$) to[bend right] node[auto,swap]{\texttt{ba}} ($(q2)+(-150:0.5)$);
\draw[Latex-] ($(qa)+(0:0.5)$) to[bend right] node[auto,swap]{\texttt{b}} ($(q2)+(270:0.5)$);
\draw[Latex-] ($(q2)+(-30:0.5)$) arc[y radius=0.25,x radius=0.5,start angle=-105,end angle=105] node[midway,auto,swap]{\texttt{a|(bb)$^*$}};
\draw[-Latex] ($(q1)+(-120:0.5)$) to[] node[auto,swap]{$\varepsilon$} ($(qa)+(60:0.5)$);
\end{tikzpicture}
\end{center}
        \item Tomando $q_{\text{rip}}=2$.
        \begin{center}
\begin{tikzpicture}[rotate=120]
% Nodes
\node (qs) at (60:2.5) []{$q_s$};
\node (q1) at (0,0) []{$1$};
\node (qa) at (-120:5) []{$q_a$};
% Circles
\draw[] (qs) circle[radius=0.5];
\draw[] (q1) circle[radius=0.5];
\draw[] (qa) circle[radius=0.35] circle[radius=0.5];
% Edges
\draw[Latex-] ($(qs)+(60:0.5)$) to[] +(60:0.5);
\draw[-Latex] ($(qs)+(-120:0.5)$) to[] node[auto,swap]{$\varepsilon$} ($(q1)+(60:0.5)$);
\draw[-Latex] ($(q1)+(-120:0.5)$) to[] node[auto,swap]{$\varepsilon$\texttt{|((a|b)(a|(bb)$^*$)$^*$b)}} ($(qa)+(60:0.5)$);
\draw[-Latex,rotate=-120] ($(q1)+(60:0.5)$) arc[x radius=0.25,y radius=0.5,start angle=0,end angle=180] node[midway,auto,swap]{\texttt{(a|b)(a|(bb)$^*$)$^*$(ba)}};
\end{tikzpicture}
\end{center}
        \item Tomando $q_{\text{rip}}=1$.
        \begin{center}
\begin{tikzpicture}
% Nodes
\node (qs) at (0,0) []{$q_s$};
\node (qa) at (9,0) []{$q_a$};
% Circles
\draw[] (qs) circle[radius=0.5];
\draw[] (qa) circle[radius=0.35] circle[radius=0.5];
% Edges
\draw[Latex-] ($(qs)+(180:0.5)$) to[] +(180:0.5);
\draw[-Latex] ($(qs)+(0:0.5)$) to[] node[auto] {$\varepsilon$(\texttt{(a|b)(a|(bb)$^*$)$^*$(ba)})$^*$($\varepsilon$\texttt{|((a|b)(a|(bb)$^*$)$^*$b)})} ($(qa)+(180:0.5)$);
\end{tikzpicture}
\end{center}
    \end{itemize}
\end{enumerate}
\end{document}
